\chapter{Les plugins indispensables}

Soyons clair, \vim sans ses plugins, c'est comme Milan sans Rémo\sidenote{\copyright François Corbier - Sans ma barbe - \url{http://www.bide-et-musique.com/song/149.html}} : ça ne rime à rien. C'est grâce aux plugins que \vim va pouvoir pleinement exprimer toute sa puissance et vous élever à un autre niveau de productivité. Vous n'avez pas besoin d'en avoir des mille et des cents, mais quelques uns savamment choisis devraient faire l'affaire.

Qu'on ne se méprenne pas, \vim peut bien sûr s'utiliser sans plugins. Il peut d'ailleurs s'avérer utile de savoir faire les manipulations de base sans avoir besoin d'installer de plugin, car c'est souvent le cas sur des serveurs : il n'y a aucun plugin d'installé. Dans ce cas là, savoir ouvrir, sauvegarder sous, passer d'un fichier à l'autre avec les commandes de \vim par défaut peut vous sauver la mise. En revanche, dans votre travail quotidien de rédaction ou de code, les plugins sont indispensables pour pleinement tirer partie de \vim.

\section{Naviguer sur le disque et entre les fichiers}

Nous avons déjà vu NerdTree plus haut qui permettait d'avoir un explorateur de projet dans une fenêtre latérale de \vim. Le problème de ce plugin est qu'il n'est pas fait pour être utilisé au clavier. Certes vous pouvez utiliser le clavier, mais il ne sera pas aussi efficace que les plugins pensés uniquement pour une utilisation au clavier.

En effet, le premier plugin que j'installe partout où j'ai à utiliser \vim, c'est \emph{Lusty Explorer}\sidenote{\url{http://www.vim.org/scripts/script.php?script_id=1890}}.
