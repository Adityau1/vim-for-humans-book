\chapter{Les plugins indispensables}

Soyons clair, \vim sans ses plugins, c'est comme Milan sans Rémo\sidenote{\copyright François Corbier - Sans ma barbe - \url{http://www.bide-et-musique.com/song/149.html}} : ça ne rime à rien. C'est grâce aux plugins que \vim va pouvoir pleinement exprimer toute sa puissance et vous élever à un autre niveau de productivité. Vous n'avez pas besoin d'en avoir des mille et des cents, mais quelques uns savamment choisis devraient faire l'affaire.

Qu'on ne se méprenne pas, \vim peut bien sûr s'utiliser sans plugins. Il peut d'ailleurs s'avérer utile de savoir faire les manipulations de base sans avoir besoin d'installer de plugin, car c'est souvent le cas sur des serveurs : il n'y a aucun plugin d'installé. Dans ce cas là, savoir ouvrir, sauvegarder sous, passer d'un fichier à l'autre avec les commandes de \vim par défaut peut vous sauver la mise. En revanche, dans votre travail quotidien de rédaction ou de code, les plugins sont indispensables pour pleinement tirer partie de \vim.
