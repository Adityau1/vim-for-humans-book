\documentclass{tufte-book}
\usepackage[french]{babel}
\usepackage[utf8]{inputenc}

\hypersetup{colorlinks}% uncomment this line if you prefer colored hyperlinks (e.g., for onscreen viewing)

%%
% Book metadata
\title{Vim - Le guide pour les êtres humains\thanks{Merci à la communauté Vim.}}
\author[Vincent Jousse]{Vincent\ Jousse}
\publisher{Vincent Jousse}

%%
% If they're installed, use Bergamo and Chantilly from www.fontsite.com.
% They're clones of Bembo and Gill Sans, respectively.
%\IfFileExists{bergamo.sty}{\usepackage[osf]{bergamo}}{}% Bembo
%\IfFileExists{chantill.sty}{\usepackage{chantill}}{}% Gill Sans

%\usepackage{microtype}

%%
% Just some sample text
\usepackage{lipsum}

%%
% For nicely typeset tabular material
\usepackage{booktabs}

%%
% For graphics / images
\usepackage{graphicx}
\setkeys{Gin}{width=\linewidth,totalheight=\textheight,keepaspectratio}
\graphicspath{{graphics/}}

% The fancyvrb package lets us customize the formatting of verbatim
% environments.  We use a slightly smaller font.
\usepackage{fancyvrb}
\fvset{fontsize=\normalsize}

%%
% Prints argument within hanging parentheses (i.e., parentheses that take
% up no horizontal space).  Useful in tabular environments.
\newcommand{\hangp}[1]{\makebox[0pt][r]{(}#1\makebox[0pt][l]{)}}

%%
% Prints an asterisk that takes up no horizontal space.
% Useful in tabular environments.
\newcommand{\hangstar}{\makebox[0pt][l]{*}}

%%
% Prints a trailing space in a smart way.
\usepackage{xspace}

% Format code using pygment
\usepackage{minted}
\usemintedstyle{solarized}
%\definecolor{bg}{RGB}{0,43,54}
%Light bg
\definecolor{bg}{RGB}{253,246,227}

%%
% Some shortcuts for Tufte's book titles.  The lowercase commands will
% produce the initials of the book title in italics.  The all-caps commands
% will print out the full title of the book in italics.
\newcommand{\vdqi}{\textit{VDQI}\xspace}
\newcommand{\ei}{\textit{EI}\xspace}
\newcommand{\ve}{\textit{VE}\xspace}
\newcommand{\be}{\textit{BE}\xspace}
\newcommand{\VDQI}{\textit{The Visual Display of Quantitative Information}\xspace}
\newcommand{\EI}{\textit{Envisioning Information}\xspace}
\newcommand{\VE}{\textit{Visual Explanations}\xspace}
\newcommand{\BE}{\textit{Beautiful Evidence}\xspace}

\newcommand{\TL}{Tufte-\LaTeX\xspace}

% Prints the month name (e.g., January) and the year (e.g., 2008)
\newcommand{\monthyear}{%
  \ifcase\month\or January\or February\or March\or April\or May\or June\or
  July\or August\or September\or October\or November\or
  December\fi\space\number\year
}

% Prints an epigraph and speaker in sans serif, all-caps type.
\newcommand{\openepigraph}[2]{%
  %\sffamily\fontsize{14}{16}\selectfont
  \begin{fullwidth}
  \sffamily\large
  \begin{doublespace}
  \noindent\allcaps{#1}\\% epigraph
  \noindent\allcaps{#2}% author
  \end{doublespace}
  \end{fullwidth}
}

% Inserts a blank page
\newcommand{\blankpage}{\newpage\hbox{}\thispagestyle{empty}\newpage}

\usepackage{units}

% Typesets the font size, leading, and measure in the form of 10/12x26 pc.
\newcommand{\measure}[3]{#1/#2$\times$\unit[#3]{pc}}

% Macros for typesetting the documentation
\newcommand{\hlred}[1]{\textcolor{Maroon}{#1}}% prints in red
\newcommand{\hangleft}[1]{\makebox[0pt][r]{#1}}
\newcommand{\hairsp}{\hspace{1pt}}% hair space
\newcommand{\hquad}{\hskip0.5em\relax}% half quad space
\newcommand{\TODO}{\textcolor{red}{\bf TODO!}\xspace}
\newcommand{\ie}{\textit{i.\hairsp{}e.}\xspace}
\newcommand{\eg}{\textit{e.\hairsp{}g.}\xspace}
\newcommand{\na}{\quad--}% used in tables for N/A cells
\providecommand{\XeLaTeX}{X\lower.5ex\hbox{\kern-0.15em\reflectbox{E}}\kern-0.1em\LaTeX}
\newcommand{\tXeLaTeX}{\XeLaTeX\index{XeLaTeX@\protect\XeLaTeX}}
% \index{\texttt{\textbackslash xyz}@\hangleft{\texttt{\textbackslash}}\texttt{xyz}}
\newcommand{\tuftebs}{\symbol{'134}}% a backslash in tt type in OT1/T1
\newcommand{\doccmdnoindex}[2][]{\texttt{\tuftebs#2}}% command name -- adds backslash automatically (and doesn't add cmd to the index)
\newcommand{\doccmddef}[2][]{%
  \hlred{\texttt{\tuftebs#2}}\label{cmd:#2}%
  \ifthenelse{\isempty{#1}}%
    {% add the command to the index
      \index{#2 command@\protect\hangleft{\texttt{\tuftebs}}\texttt{#2}}% command name
    }%
    {% add the command and package to the index
      \index{#2 command@\protect\hangleft{\texttt{\tuftebs}}\texttt{#2} (\texttt{#1} package)}% command name
      \index{#1 package@\texttt{#1} package}\index{packages!#1@\texttt{#1}}% package name
    }%
}% command name -- adds backslash automatically
\newcommand{\doccmd}[2][]{%
  \texttt{\tuftebs#2}%
  \ifthenelse{\isempty{#1}}%
    {% add the command to the index
      \index{#2 command@\protect\hangleft{\texttt{\tuftebs}}\texttt{#2}}% command name
    }%
    {% add the command and package to the index
      \index{#2 command@\protect\hangleft{\texttt{\tuftebs}}\texttt{#2} (\texttt{#1} package)}% command name
      \index{#1 package@\texttt{#1} package}\index{packages!#1@\texttt{#1}}% package name
    }%
}% command name -- adds backslash automatically
\newcommand{\docopt}[1]{\ensuremath{\langle}\textrm{\textit{#1}}\ensuremath{\rangle}}% optional command argument
\newcommand{\docarg}[1]{\textrm{\textit{#1}}}% (required) command argument
\newenvironment{docspec}{\begin{quotation}\ttfamily\parskip0pt\parindent0pt\ignorespaces}{\end{quotation}}% command specification environment
\newcommand{\docenv}[1]{\texttt{#1}\index{#1 environment@\texttt{#1} environment}\index{environments!#1@\texttt{#1}}}% environment name
\newcommand{\docenvdef}[1]{\hlred{\texttt{#1}}\label{env:#1}\index{#1 environment@\texttt{#1} environment}\index{environments!#1@\texttt{#1}}}% environment name
\newcommand{\docpkg}[1]{\texttt{#1}\index{#1 package@\texttt{#1} package}\index{packages!#1@\texttt{#1}}}% package name
\newcommand{\doccls}[1]{\texttt{#1}}% document class name
\newcommand{\docclsopt}[1]{\texttt{#1}\index{#1 class option@\texttt{#1} class option}\index{class options!#1@\texttt{#1}}}% document class option name
\newcommand{\docclsoptdef}[1]{\hlred{\texttt{#1}}\label{clsopt:#1}\index{#1 class option@\texttt{#1} class option}\index{class options!#1@\texttt{#1}}}% document class option name defined
\newcommand{\docmsg}[2]{\bigskip\begin{fullwidth}\noindent\ttfamily#1\end{fullwidth}\medskip\par\noindent#2}
\newcommand{\docfilehook}[2]{\texttt{#1}\index{file hooks!#2}\index{#1@\texttt{#1}}}
\newcommand{\doccounter}[1]{\texttt{#1}\index{#1 counter@\texttt{#1} counter}}

% Keys shortcuts
\newcommand{\tesc}{\hlred{Esc} (\hlred{Échap})\xspace}
\newcommand{\ttesc}{la touche \tesc}
\newcommand{\tshift}{\hlred{Shift}\xspace}
\newcommand{\ttshift}{la touche \tshift}
\newcommand{\ti}{\hlred{i}\xspace}
\newcommand{\tti}{la touche \ti}
\newcommand{\ttm}{la touche \hlred{m}\xspace}
\newcommand{\tto}{la touche \hlred{o}\xspace}
\newcommand{\tp}{\hlred{p}\xspace}
\newcommand{\ttp}{la touche \tp}
\newcommand{\tv}{\hlred{v}\xspace}
\newcommand{\ttv}{la touche \tv}
\newcommand{\ty}{\hlred{y}\xspace}
\newcommand{\tty}{la touche \ty}

\newcommand{\vimscmd}[1]{\fcolorbox{black}{bg}{\hlred{\Verb|{\footnotesize #1}|}}}
\newcommand{\vimcmd}[1]{\fcolorbox{black}{bg}{\hlred{\Verb|#1|}}}

\newcommand{\scmd}[1]{\fcolorbox{black}{white}{\hlred{\Verb|{\footnotesize #1}|}}}
\newcommand{\ncmd}[1]{\fcolorbox{black}{white}{\hlred{\Verb|#1|}}}

\newcommand{\vim}{\emph{Vim}\xspace}
\newcommand{\vimrc}{\emph{.vimrc}\xspace}


% Generates the index
\usepackage{makeidx}
\makeindex


\begin{document}

% Front matter
\frontmatter

% r.1 blank page
\blankpage

% r.3 full title page
\maketitle


% v.4 copyright page
\newpage
\begin{fullwidth}
~\vfill
\thispagestyle{empty}
\setlength{\parindent}{0pt}
\setlength{\parskip}{\baselineskip}
Copyright \copyright\ \the\year\ \thanklessauthor

\par\smallcaps{Publié par \thanklesspublisher}

\par\smallcaps{Style \LaTeX{} \url{http://tufte-latex.googlecode.com}}

\par Si vous n'avez pas payé cette copie, bah tant pis pour moi ;)

\end{fullwidth}

% r.5 contents
\tableofcontents

\listoffigures

\listoftables

% r.7 dedication
\cleardoublepage
~\vfill
\begin{doublespace}
\noindent\fontsize{18}{22}\selectfont\itshape
\nohyphenation
Merci à ma femme et mes enfants qui ont permis à ce guide de voir le jour.
\end{doublespace}
\vfill
\vfill

% r.9 introduction
\cleardoublepage
\chapter*{Introduction}

\section{Pourquoi ?}

\begin{description}
    \item[Top of the pops] \hfill \\ \emph{Vim} est un des meilleurs éditeurs de texte au monde, depuis 1991\sidenote{Emacs est bien aussi hein, mais c'est pas le sujet là. Je vous vois venir avec vos trolls.}.
    \item[Pour la vie] \hfill \\ \emph{Vim} s'apprend une fois et s'utilise partout, pour toujours\sidenote{Dans un monde où les technologies/langages changent tout le temps, c'est rare d'investir sur le long terme.}.
    \item[Enlarge votre productivité] \hfill \\ \emph{Vim} est le meilleur rapport temps investi / gain de productivité pour toute personne éditant du texte.
\end{description}

Et pourtant, \emph{Vim} reste difficile à apprendre. Non pas qu'il soit plus compliqué qu'autre chose, non pas que vous ne soyez pas à la hauteur, mais plutôt à cause d'un manque de pédagogie des différentes documentations.

Marre d'attendre la release de TextMate 2 ? D'essayer le n-ième éditeur à la mode et de devoir tout réapprendre et ce jusqu'à la prochaine mode ? Marre de devoir changer d'éditeur quand vous passez de votre Mac, à votre Linux, à votre serveur ? Vous aussi, rejoignez la communauté des gens contents de leur éditeur de texte. Le changement, c'est maintenant.

\section{Pour qui ?}

Toute personne étant amenée à produire du texte (code, livre, rapports, présentations, ...) de manière régulière. Les développeurs sont bien sur une cible privilégiée, mais pas uniquement.

Par exemple vous êtes :
\begin{description}
    \item[Étudiant] Si vous voulez faire bien sur un CV, c'est un must (en plus d'être un attrape geekette en puissance\footnote{À confirmer.}). Vous aurez un outil unique pour écrire tout ce que vous avez à écrire (et que vous pourrez réutiliser tout au long de votre carrière) : vos rapports en \LaTeX, vos présentations, votre code (si vous avez besoin d'OpenOffice ou de Word pour écrire vos rapports, il est temps de changer d'outil et d'utiliser \LaTeX).
    \item[Prof] Il est temps de montrer l'exemple et d'apprendre à vos étudiants à bien utiliser un des outils qui leur servira à vie, bien plus qu'un quelconque langage de programmation.
    \item[Codeur] Investir dans votre outil de tous les jours est indispensable. Quitte à apprendre des raccourcis claviers, autant le faire de manière utile. Si cet investissement est encore rentable dans 10 ans, c'est juste l'investissement parfait, c'est \emph{Vim}.
    \item[Administrateur système Unix] Si vous utilisez emacs vous êtes pardonnable, si vous utilisez nano/pico vous êtes pendable, sinon il est grand temps de s'y mettre les gars, c'est un des cas d'utilisation parfait (un éditeur de texte surpuissant ne nécessitant pas d'interface graphique).
    \item[Écrivain] Si vous écrivez en markdown/RST/WikiMarkup ou en \LaTeX, \emph{Vim} vous fera gagner beaucoup de temps. Vous ne pourrez plus repasser à un autre éditeur, ou vous voudrez le "Vimifier" à tout prix.
\end{description}

Faites moi confiance, je suis passé et repassé par ces 5 rôles, mon meilleur investissement a toujours été \emph{Vim}, et de loin.

\section{Ce que vous apprendrez (entre autres choses)}

\begin{itemize}
    \item Comment utiliser \emph{Vim} comme un éditeur "normal" d'abord (vous savez, ceux qui permettent d'ouvrir des fichiers, de cliquer avec la souris et de faire des recherches dans les fichiers du projet ...). En somme, la démystification de \emph{Vim} qui vous permettra d'aller plus loin.
    \item Comment passer de l'édition de texte classique à la puissance de \emph{Vim}, petit à petit (c'est là que l'addiction commence).
    \item Comment vous passer de la souris et pourquoi c'est la meilleure chose qu'il puisse vous arriver quand vous programmez/tapez du texte.
    \item Comment vous pouvez facilement déduire les "raccourcis claviers" avec quelques règles simples.
\end{itemize}

Si je devais le résumer en une phrase : puisque vous vous considérez comme {\bf un artiste, passez du temps à apprendre votre outil}, comme un artiste, une bonne fois pour toute.

\section{Ce que vous n'apprendrez pas (entre autres choses)}

\begin{itemize}
    \item Vous n'apprendrez pas comment installer/configurer {\em Vim} pour Windows. Pas que ce ne soit pas faisable, mais je n'ai que très peu de connaissances sous Windows. Ça viendra peut-être, mais pas tout de suite.
    \item Vous n'apprendrez pas comment utiliser \emph{Vi} (notez l'absence du "m"). Je vais vous apprendre à être productif pour coder/produire du texte avec \emph{Vim}, pas à faire le beau devant les copains avec \emph{Vi}. Pour ceux qui ne suivent pas, \emph{Vi} est "l'ancêtre de \emph{ViM} (qui veut dire \emph{Vi} - \emph{IMproved}, \emph{Vi} amélioré)" et est installé par défaut sur tous les Unix (même sur votre Mac OS X).
    \item Vous n'apprendrez pas à connaitre les entrailles de \emph{Vim} par c\oe ur : ce n'est pas une référence, mais un guide utile et pragmatique.
    \item Vous n'apprendrez pas comment modifier votre \emph{Vim} parce que vous préférez le rouge au bleu : je vous ferai utiliser le thème [solarized](http://ethanschoonover.com/solarized), il est juste parfait pour travailler.
\end{itemize}

\section{Le plus dur, c'est de commencer (et de continuer à commencer)}

Alors, prêt pour l'aventure ? Prêt à sacrifier une heure pour débuter avec \emph{Vim}, une semaine pour devenir familier avec la bête, et le reste de votre vie pour vous féliciter de votre choix ? Alors c'est parti !


%%
% Start the main matter (normal chapters)
\mainmatter



\printindex

\end{document}
