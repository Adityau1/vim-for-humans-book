\documentclass[notoc]{tufte-book}
\usepackage[french]{babel}
\usepackage[utf8]{inputenc}

\hypersetup{colorlinks}% uncomment this line if you prefer colored hyperlinks (e.g., for onscreen viewing)

%%
% Book metadata
\title{Vim pour les humains\thanks{Merci à la communauté Vim.}}
\author[Vincent Jousse]{Vincent\ Jousse}
\publisher{Livre à prix libre}

%%
% If they're installed, use Bergamo and Chantilly from www.fontsite.com.
% They're clones of Bembo and Gill Sans, respectively.
%\IfFileExists{bergamo.sty}{\usepackage[osf]{bergamo}}{}% Bembo
%\IfFileExists{chantill.sty}{\usepackage{chantill}}{}% Gill Sans

%\usepackage{microtype}

%%
% Just some sample text
\usepackage{lipsum}

%%
% For nicely typeset tabular material
\usepackage{booktabs}

%%
% To manage table width
\usepackage{tabularx}

%%
% For graphics / images
\usepackage{graphicx}
\setkeys{Gin}{width=\linewidth,totalheight=\textheight,keepaspectratio}
\graphicspath{{graphics/}}

% The fancyvrb package lets us customize the formatting of verbatim
% environments.  We use a slightly smaller font.
\usepackage{fancyvrb}
\fvset{fontsize=\normalsize}

%%
% Prints argument within hanging parentheses (i.e., parentheses that take
% up no horizontal space).  Useful in tabular environments.
\newcommand{\hangp}[1]{\makebox[0pt][r]{(}#1\makebox[0pt][l]{)}}

%%
% Prints an asterisk that takes up no horizontal space.
% Useful in tabular environments.
\newcommand{\hangstar}{\makebox[0pt][l]{*}}

%%
% Prints a trailing space in a smart way.
\usepackage{xspace}

% Format code using pygment
\usepackage{minted}
\usemintedstyle{vim}
%\definecolor{bg}{RGB}{0,43,54}
%Light bg
\definecolor{bg}{RGB}{253,246,227}

%%
% Some shortcuts for Tufte's book titles.  The lowercase commands will
% produce the initials of the book title in italics.  The all-caps commands
% will print out the full title of the book in italics.
\newcommand{\vdqi}{\textit{VDQI}\xspace}
\newcommand{\ei}{\textit{EI}\xspace}
\newcommand{\ve}{\textit{VE}\xspace}
\newcommand{\be}{\textit{BE}\xspace}
\newcommand{\VDQI}{\textit{The Visual Display of Quantitative Information}\xspace}
\newcommand{\EI}{\textit{Envisioning Information}\xspace}
\newcommand{\VE}{\textit{Visual Explanations}\xspace}
\newcommand{\BE}{\textit{Beautiful Evidence}\xspace}

\newcommand{\TL}{Tufte-\LaTeX\xspace}

% Prints the month name (e.g., January) and the year (e.g., 2008)
\newcommand{\monthyear}{%
  \ifcase\month\or January\or February\or March\or April\or May\or June\or
  July\or August\or September\or October\or November\or
  December\fi\space\number\year
}

% Prints an epigraph and speaker in sans serif, all-caps type.
\newcommand{\openepigraph}[2]{%
  %\sffamily\fontsize{14}{16}\selectfont
  \begin{fullwidth}
  \sffamily\large
  \begin{doublespace}
  \noindent\allcaps{#1}\\% epigraph
  \noindent\allcaps{#2}% author
  \end{doublespace}
  \end{fullwidth}
}

% Inserts a blank page
\newcommand{\blankpage}{\newpage\hbox{}\thispagestyle{empty}\newpage}

\usepackage{units}

% Typesets the font size, leading, and measure in the form of 10/12x26 pc.
\newcommand{\measure}[3]{#1/#2$\times$\unit[#3]{pc}}

% Macros for typesetting the documentation
\newcommand{\hlred}[1]{\textcolor{Maroon}{#1}}% prints in red
\newcommand{\hangleft}[1]{\makebox[0pt][r]{#1}}
\newcommand{\hairsp}{\hspace{1pt}}% hair space
\newcommand{\hquad}{\hskip0.5em\relax}% half quad space
\newcommand{\TODO}{\textcolor{red}{\bf TODO!}\xspace}
\newcommand{\ie}{\textit{i.\hairsp{}e.}\xspace}
\newcommand{\eg}{\textit{e.\hairsp{}g.}\xspace}
\newcommand{\na}{\quad--}% used in tables for N/A cells
\providecommand{\XeLaTeX}{X\lower.5ex\hbox{\kern-0.15em\reflectbox{E}}\kern-0.1em\LaTeX}
\newcommand{\tXeLaTeX}{\XeLaTeX\index{XeLaTeX@\protect\XeLaTeX}}
% \index{\texttt{\textbackslash xyz}@\hangleft{\texttt{\textbackslash}}\texttt{xyz}}
\newcommand{\tuftebs}{\symbol{'134}}% a backslash in tt type in OT1/T1
\newcommand{\doccmdnoindex}[2][]{\texttt{\tuftebs#2}}% command name -- adds backslash automatically (and doesn't add cmd to the index)
\newcommand{\doccmddef}[2][]{%
  \hlred{\texttt{\tuftebs#2}}\label{cmd:#2}%
  \ifthenelse{\isempty{#1}}%
    {% add the command to the index
      \index{#2 command@\protect\hangleft{\texttt{\tuftebs}}\texttt{#2}}% command name
    }%
    {% add the command and package to the index
      \index{#2 command@\protect\hangleft{\texttt{\tuftebs}}\texttt{#2} (\texttt{#1} package)}% command name
      \index{#1 package@\texttt{#1} package}\index{packages!#1@\texttt{#1}}% package name
    }%
}% command name -- adds backslash automatically
\newcommand{\doccmd}[2][]{%
  \texttt{\tuftebs#2}%
  \ifthenelse{\isempty{#1}}%
    {% add the command to the index
      \index{#2 command@\protect\hangleft{\texttt{\tuftebs}}\texttt{#2}}% command name
    }%
    {% add the command and package to the index
      \index{#2 command@\protect\hangleft{\texttt{\tuftebs}}\texttt{#2} (\texttt{#1} package)}% command name
      \index{#1 package@\texttt{#1} package}\index{packages!#1@\texttt{#1}}% package name
    }%
}% command name -- adds backslash automatically
\newcommand{\docopt}[1]{\ensuremath{\langle}\textrm{\textit{#1}}\ensuremath{\rangle}}% optional command argument
\newcommand{\docarg}[1]{\textrm{\textit{#1}}}% (required) command argument
\newenvironment{docspec}{\begin{quotation}\ttfamily\parskip0pt\parindent0pt\ignorespaces}{\end{quotation}}% command specification environment
\newcommand{\docenv}[1]{\texttt{#1}\index{#1 environment@\texttt{#1} environment}\index{environments!#1@\texttt{#1}}}% environment name
\newcommand{\docenvdef}[1]{\hlred{\texttt{#1}}\label{env:#1}\index{#1 environment@\texttt{#1} environment}\index{environments!#1@\texttt{#1}}}% environment name
\newcommand{\docpkg}[1]{\texttt{#1}\index{#1 package@\texttt{#1} package}\index{packages!#1@\texttt{#1}}}% package name
\newcommand{\doccls}[1]{\texttt{#1}}% document class name
\newcommand{\docclsopt}[1]{\texttt{#1}\index{#1 class option@\texttt{#1} class option}\index{class options!#1@\texttt{#1}}}% document class option name
\newcommand{\docclsoptdef}[1]{\hlred{\texttt{#1}}\label{clsopt:#1}\index{#1 class option@\texttt{#1} class option}\index{class options!#1@\texttt{#1}}}% document class option name defined
\newcommand{\docmsg}[2]{\bigskip\begin{fullwidth}\noindent\ttfamily#1\end{fullwidth}\medskip\par\noindent#2}
\newcommand{\docfilehook}[2]{\texttt{#1}\index{file hooks!#2}\index{#1@\texttt{#1}}}
\newcommand{\doccounter}[1]{\texttt{#1}\index{#1 counter@\texttt{#1} counter}}

% Keys shortcuts
\newcommand{\tapos}{\hlred{\Verb|'|}}
\newcommand{\ttapos}{la touche \tapos}
\newcommand{\tcolon}{\hlred{\Verb|:|}}
\newcommand{\ttcolon}{la touche \tcolon}
\newcommand{\tcomma}{\hlred{\Verb|, (la virgule)|}}
\newcommand{\tctrl}{\hlred{\Verb|Ctrl|} (\hlred{\Verb|Control|})\xspace}
\newcommand{\ttctrl}{la touche \tctrl}
\newcommand{\tdollar}{\hlred{\Verb|\$|}\xspace}
\newcommand{\ttdollar}{la touche \tdollar}
\newcommand{\tenter}{\hlred{\Verb|Entrée|}\xspace}
\newcommand{\ttenter}{la touche \tenter}
\newcommand{\tesc}{\hlred{\Verb|Esc|} (\hlred{\Verb|Échap|})\xspace}
\newcommand{\ttesc}{la touche \tesc}
\newcommand{\that}{\hlred{\Verb|\^{ }|}\xspace}
\newcommand{\tthat}{la touche \that}
\newcommand{\tleader}{\hlred{\Verb|<Leader>|}}
\newcommand{\tshift}{\hlred{\Verb|Shift|}\xspace}
\newcommand{\ttshift}{la touche \tshift}
\newcommand{\tsemicolon}{\hlred{\Verb|;|}\xspace}
\newcommand{\ttsemicolon}{la touche \tsemicolon}
\newcommand{\tsharp}{\hlred{\Verb|\#|}\xspace}
\newcommand{\ttsharp}{la touche \tsharp}
\newcommand{\tslash}{\hlred{\Verb|/|}\xspace}
\newcommand{\ttslash}{la touche \tslash}
\newcommand{\tstar}{\hlred{\Verb|*|}\xspace}
\newcommand{\ttstar}{la touche \tstar}
\newcommand{\tzero}{\hlred{\Verb|0|}\xspace}
\newcommand{\ttzero}{la touche \tzero}
\newcommand{\tth}{la touche \hlred{\Verb|h|}\xspace}
\newcommand{\ta}{\hlred{\Verb|a|}\xspace}
\newcommand{\tta}{la touche \ta}
\newcommand{\tA}{\hlred{\Verb|A|}\xspace}
\newcommand{\ttA}{la touche \tA}
\newcommand{\tb}{\hlred{\Verb|b|}\xspace}
\newcommand{\ttb}{la touche \tb}
\newcommand{\tc}{\hlred{\Verb|c|}}
\newcommand{\ttc}{la touche \tc}
\newcommand{\td}{\hlred{\Verb|d|}}
\newcommand{\ttd}{la touche \td}
\newcommand{\te}{\hlred{\Verb|e|}\xspace}
\newcommand{\tte}{la touche \te}
\newcommand{\tf}{\hlred{\Verb|f|}\xspace}
\newcommand{\ttf}{la touche \tf}
\newcommand{\tg}{\hlred{\Verb|g|}}
\newcommand{\ttg}{la touche \tg}
\newcommand{\tG}{\hlred{\Verb|G|}}
\newcommand{\ttG}{la touche \tG}
\newcommand{\ti}{\hlred{\Verb|i|}\xspace}
\newcommand{\tti}{la touche \ti}
\newcommand{\tI}{\hlred{\Verb|I|}\xspace}
\newcommand{\ttI}{la touche \tI}
\newcommand{\tj}{\hlred{\Verb|j|}\xspace}
\newcommand{\ttj}{la touche \tj}
\newcommand{\tk}{\hlred{\Verb|k|}\xspace}
\newcommand{\ttk}{la touche \tk}
\newcommand{\tl}{\hlred{\Verb|l|}\xspace}
\newcommand{\ttl}{la touche \tl}
\newcommand{\tm}{\hlred{\Verb|m|}}
\newcommand{\ttm}{la touche \tm}
\newcommand{\tn}{\hlred{\Verb|n|}}
\newcommand{\ttn}{la touche \tn}
\newcommand{\tN}{\hlred{\Verb|N|}}
\newcommand{\ttN}{la touche \tN}
\newcommand{\kto}{\hlred{\Verb|o|}\xspace}
\newcommand{\tto}{la touche \hlred{\Verb|o|}\xspace}
\newcommand{\tO}{\hlred{\Verb|O|}\xspace}
\newcommand{\ttO}{la touche \hlred{\Verb|O|}\xspace}
\newcommand{\tp}{\hlred{\Verb|p|}\xspace}
\newcommand{\ttp}{la touche \tp}
\newcommand{\tP}{\hlred{\Verb|P|}\xspace}
\newcommand{\ttP}{la touche \tP}
\newcommand{\tr}{\hlred{\Verb|r|}\xspace}
\newcommand{\ttr}{la touche \tr}
\newcommand{\tu}{\hlred{\Verb|u|}\xspace}
\newcommand{\ttu}{la touche \tu}
\newcommand{\tv}{\hlred{\Verb|v|}\xspace}
\newcommand{\ttv}{la touche \tv}
\newcommand{\tV}{\hlred{\Verb|V|}\xspace}
\newcommand{\ttV}{la touche \tV}
\newcommand{\tw}{\hlred{\Verb|w|}\xspace}
\newcommand{\ttw}{la touche \tw}
\newcommand{\tx}{\hlred{\Verb|x|}\xspace}
\newcommand{\ttx}{la touche \tx}
\newcommand{\tX}{\hlred{\Verb|X|}\xspace}
\newcommand{\ttX}{la touche \tX}
\newcommand{\ty}{\hlred{\Verb|y|}}
\newcommand{\tty}{la touche \ty}

\newcommand{\vimscmd}[1]{\fcolorbox{black}{bg}{\hlred{\Verb|{\footnotesize #1}|}}}
\newcommand{\vimcmd}[1]{\fcolorbox{black}{bg}{\hlred{\Verb|#1|}}}

\newcommand{\scmd}[1]{\fcolorbox{black}{white}{\hlred{\Verb|{\footnotesize #1}|}}}
\newcommand{\ncmd}[1]{\fcolorbox{black}{white}{\hlred{\Verb|#1|}}}
\newcommand{\vimshortcut}[1]{\fcolorbox{white}{bg}{\hlred{\Verb|#1|}}}

\newcommand{\vim}{\emph{Vim}\xspace}
\newcommand{\vimrc}{\emph{.vimrc}\xspace}
\newcommand{\dotvim}{\emph{.vim/}\xspace}

\usepackage{fancyhdr}
\pagestyle{fancy}

\makeatletter% so we can use @ symbols in macro names
\newcommand{\myfancyhf}{%
\fancyhead{}% clear previous contents of running heads/feet
%\fancyhead{} % clear all footer fields
\fancyhead[LE,RO]{\thepage}           % page number in "outer" position of footer line
\fancyhead[RE,LO]{\emph{Vim pour les humains — \url{http://vimebook.com}}} % other info in "inner" position of footer line

}
\g@addto@macro{\frontmatter}{\myfancyhf}% calls \myfancyhf any time \frontmatter is called
\g@addto@macro{\mainmatter}{\myfancyhf}% likewise for \mainmatter \makeatother% restores the original meaning of @

% Generates the index
\usepackage{makeidx}
\makeindex
